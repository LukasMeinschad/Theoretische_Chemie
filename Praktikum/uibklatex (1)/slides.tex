%% Begin slides template file
\documentclass[11pt,t,usepdftitle=false,aspectratio=169]{beamer}
%% ------------------------------------------------------------------
%% - aspectratio=43: Set paper aspect ratio to 4:3.
%% - aspectratio=169: Set paper aspect ratio to 16:9.
%% ------------------------------------------------------------------

\usetheme[nototalframenumber,foot,logo]{uibk}
%% ------------------------------------------------------------------
%% - foot: Add a footer line for conference name and date.
%% - logo: Add the university logo in the footer (only if 'foot' set).
%% - bigfoot/sasquatch: Larger font size in footer.
%% - nototalslidenumber: Hide the total number of slides (only if 'foot' set)
%% - license: Add CC-BY license symbol to title slide (e.g., for conference uploads)
%%   (TODO: At the moment no other licenses are supported.)
%% - licenseall: Add CC-BY license symbol to all subsequent slides slides
%% - url: use \url{} rather than \href{} on the title page
%% - nosectiontitlepage: switches off the behaviour of inserting the
%%   titlepage every time a \section is called. This makes it possible to
%%   use more than one section + thanks page and a ToC off by default.
%%   If the 'nosectiontitlepage' is set you can create UIBK title slides
%%   using the command '\uibktitlepage{}' in your document to create
%%   one or multiple title slides.
%% ------------------------------------------------------------------

%% ------------------------------------------------------------------
%% The official corporate colors of the university are predefined and
%% can be used for e.g., highlighting something. Simply use
%% \color{uibkorange} or \begin{color}{uibkorange} ... \end{color}
%% Defined colors are:
%% - uibkblue, uibkbluel, uibkorange, uibkorangel, uibkgray, uibkgraym, uibkgrayl
%% The frametitle color can be easily adjusted e.g., to black with
%% \setbeamercolor{titlelike}{fg=black}
%% ------------------------------------------------------------------

%\setbeamercolor{verbcolor}{fg=uibkorange}
%% ------------------------------------------------------------------
%% Setting a highlight color for verbatim output such as from
%% the commands \pkg, \email, \file, \dataset 
%% ------------------------------------------------------------------


%% information for the title page ('short title' is the pdf-title that is shown in viewer's titlebar)
\title[Your Short Title Here]{Presentation Slides: With Your Title Here}
\subtitle{The {\LaTeX} Beamer Implementation}
\URL{www.uibk.ac.at/statistics}

\author[Gabriele \& Max Mustermann]{Gabriele Mustermann, Max Mustermann}
%('short author' is the pdf-metadata Author)
%% If multiple authors are required and the font size is too large you
%% can overrule the font size of author and url by calling:
%\setbeamerfont{author}{size*={10pt}{10pt},series=\mdseries}
%\setbeamerfont{url}{size*={10pt}{10pt},series=\mdseries}
%\URL{}
%\subtitle{}

\footertext{{\LaTeX} beamer theme}
\date{2017-07-25}

\headerimage{3}
%% ------------------------------------------------------------------
%% The theme offers four different header images based on the
%% corporate design of the university of innsbruck. Currently
%% 1, 2, 3 and 4 is allowed as input to \headerimage{...}. Default
%% or fallback is '1'.
%% ------------------------------------------------------------------

\begin{document}

%% ALTERNATIVE TITLEPAGE
%% The next block is how you add a titlepage with the 'nosectiontitlepage' option, which switches off
%% the default behavior of creating a titlepage every time a \section{} is defined.
%% Then you can use \section{} as it's originally intended, including a table of contents.
% \usebackgroundtemplate{\includegraphics[width=\paperwidth,height=\paperheight]{titlebackground.pdf}}
% \begin{frame}[plain]
%     \titlepage
% \end{frame}
% \addtocounter{framenumber}{-1}
% \usebackgroundtemplate{}

%% Table of Contents, if wanted:
%% this requires the 'nosectiontitlepage' option and setting \section{}'s as you want them to appear here.
%% Subsections and subordinates are suppressed in the .sty at the moment, search
%% for \setbeamertemplate{subsection} and replace the empty {} with whatever you want.
%% Although it's probably too much for a presentation, maybe for a lecture.
%% Please note: \maketitle allows you to render a uibk-style title page wherever needed
%% in the document even if 'nosectiontitlepage' option is set (note: \maketitle will not
%% create a new section and is therefore not included in \tableofcontents (if used).
% \maketitle
% \begin{frame}
%     \vspace*{1cm plus 1fil}
%     \tableofcontents
%     \vspace*{0cm plus 1fil}
% \end{frame}


%% this sets the first PDF bookmark and triggers generation of the title page
\section{Bookmark Title}

%% this just generates PDF bookmarks
\subsection{Overview}

%% first slide
\begin{frame}
\frametitle{Overview}

  \textbf{Content:} Aligned with the header above and the text from the university logo in the footer.

  \bigskip

  \textbf{Subsequently:} A tour of the most important functionality in the beamer theme.

\end{frame}


%% next PDF bookmark
\subsection{Introduction}


%% second slide
\begin{frame}
\frametitle{Introduction}

   
   \textbf{Itemized lists:}
   \begin{itemize}
      \item Level 1
      \begin{itemize}
         \item Level 2
         \begin{itemize}
            \item Level 3
         \end{itemize}
      \end{itemize}
   \end{itemize}

\bigskip

   \textbf{Enumerated lists:}
   \begin{enumerate}
      \item First item
      \item Second item
      \item Third item
   \end{enumerate}

\end{frame}


%% next PDF bookmark
\subsection{Formatting}

\begin{frame}[fragile]
\frametitle{Formatting}

   \textbf{Highlighting:} \textbf{bold}, \textit{italic}, {\color{uibkblue} UIBK blue}, {\color{uibkorange} UIBK orange}, \dots

   \bigskip
   
   \textbf{Predefined colors:}
   
   \smallskip
   
   \fcolorbox{black}{uibkblue}{\rule{0pt}{.6em}\rule{.5em}{0pt}} \quad blue (\verb|uibkblue|) \\[1mm]
   \fcolorbox{black}{uibkbluel}{\rule{0pt}{.6em}\rule{.5em}{0pt}} \quad light blue (\verb|uibkbluel|) \\[1mm]
   \fcolorbox{black}{uibkorange}{\rule{0pt}{.6em}\rule{.5em}{0pt}} \quad orange (\verb|uibkorange|) \\[1mm]
   \fcolorbox{black}{uibkorangel}{\rule{0pt}{.6em}\rule{.5em}{0pt}} \quad light orange (\verb|uibkorangel|) \\[1mm]
   \fcolorbox{black}{uibkgray}{\rule{0pt}{.6em}\rule{.5em}{0pt}} \quad gray (\verb|uibkgray|) \\[1mm]
   \fcolorbox{black}{uibkgraym}{\rule{0pt}{.6em}\rule{.5em}{0pt}} \quad medium gray (\verb|uibkgraym|) \\[1mm]
   \fcolorbox{black}{uibkgrayl}{\rule{0pt}{.6em}\rule{.5em}{0pt}} \quad light gray (\verb|uibkgrayl|)

\end{frame}


\begin{frame}[fragile]
\frametitle{Beamer blocks}

   \begin{block}{Block}
      This is a block.

      Uses \verb|uibkblue| and \verb|uibkbluel|.
   \end{block}

\medskip

   \begin{alertblock}{Alert block}
      This is an alert block.

      Uses \verb|uibkorange| and \verb|uibkorangel|.
   \end{alertblock}

\medskip

   \begin{exampleblock}{Example block}
      This is an example block.

      Uses \verb|uibkgraym| and \verb|uibkgrayl|.
   \end{exampleblock}

\end{frame}


\begin{frame}[fragile]
\frametitle{Good to know}

\textbf{Commands:} For code/data formatting provided by this beamer theme.

\bigskip
   
   \begin{tabular}{ll}
      \hline
      Command & Output example \\
      \hline
      \verb|\fct{...}|     & \fct{example} \\
      \verb|\class{...}|   & \class{example} \\
      \verb|\pkg{...}|     & \pkg{example} \\
      \verb|\email{...}|   & \email{example@email.org} \\
      \verb|\doi{...}|     & \doi{10.1234/example} \\
      \verb|\file{...}|    & \file{example} \\
      \verb|\dataset{...}| & \dataset{example} \\
      \hline
   \end{tabular}

\end{frame}


\subsection{Installation}

\begin{frame}[fragile]
\frametitle{Installation}

   \textbf{Presentation slides:}
   \begin{itemize}
      \item Copy both the file \file{beamerthemeuibk.sty} and the entire folder \file{\_images}.
  
      \item Either to your local working directory\dots
  
      \item \dots or to your TEXMF tree (\emph{recommended}), e.g., \file{texmf/tex/latex/uibk-beamer/}.
  
      \item Start with the file \file{slides.tex}, containing the sources for this demo
        along with many useful hints and comments in the source code.
    
      \item Compile with \texttt{xelatex} or \texttt{pdflatex}.
   \end{itemize}

\medskip

   \textbf{Posters:}
   \begin{itemize}
      \item Additionally copy \file{beamerthemeuibkposter.sty}.
      \item Use \file{poster.tex} as the starting point.
   \end{itemize}

\end{frame}


%% to show a last slide similar to the title slide: information for the last page
\title{Thank you for your attention!}
\subtitle{}
\section{Thanks}


%% appendix of 'extra' slides
\appendix

\begin{frame}
\frametitle{Appendix 1}
    This slide does not increase the total number of slides and can hold additional information
    that you may be asked about after the end of the presentation.
\end{frame}

\begin{frame}
\frametitle{Appendix 2}
    This slide does not increase the total number of slides and can hold additional information
    that you may be asked about after the end of the presentation.
\end{frame}

\end{document}

